\section{実験}
本実験の目標は,変異の際の活性化関数の慎重な選択をすることによってよりよい精度を実現することである.つまり探索後期において,ネットワークの出力が大きく異なる変異を避けることで,局所的な解を重点的に探索できるようにする.また, $ \epsilon $ の値により探索初期の収束速度の向上も期待する.

\subsection{実験環境}
以下は本実験を行うにあたり使用した環境である.

\begin{table}[h]
    \caption{実験環境}
    \centering
    \begin{tabular}{cl}
        \hline
        OS & Ubuntu 22.04.3 LTS(64bit) \\
        CPU & Core(TM) i7-12700 \\
        GPU & GeForce RTX 3080 \\
        RAM & 32.0GiB \\
        使用言語 & Python 3.8.12 \\
        \hline
    \end{tabular}
\end{table}

実験を行うにあたって,CPUの性能は個体の変異や選択に要する時間に関係する.今回12コアのうち,マスターコアに1コア,スレーブコアに11コアを割り当てている.GPUの性能は個体がタスクを得く際の適応度の測定時間に関係する.浮動小数点演算には64bitを用いることが推奨されているが,今回のタスクにおいて大きな誤差が確認できないことから,速度の早い32bitを使用する.

\subsection{実験内容}
本実験では,既存手法,区間積分差による提案手法,個体が経験した入力に対する出力差による提案手法の3つを比較する.また,提案手法については $ \epsilon $ の有無についても比較する.以下は上記の5つの異なるパラメータを表した表である.

\begin{table}[h]
    \caption{比較する実験}
    \centering
    \begin{tabular}{ccll}
        \hline
        実験番号  & 活性化関数の変更アルゴリズム & $ \epsilon $ の初期値 & $ \epsilon $ の減衰率 \\
        \hline \hline
        1 & 既存手法 & - & - \\
        2 & 区間積分差を用いた慎重な選択 & 0.1 & 1.0 \\
        3 & 区間積分差を用いた慎重な選択 & 1.0 & 0.99 \\
        4 & 個体の経験に基づく出力差を用いた慎重な選択 & 0.1 & 1.0 \\
        5 & 個体の経験に基づく出力差を用いた慎重な選択 & 1.0 & 0.99 \\
        \hline
    \end{tabular}
\end{table}

式(7)で示した通り, $ \epsilon $ の値は大きければ大きいほど,変更後の活性化関数が関数同士の距離に依存しなくなり,既存手法に近いものになる.これを世代と共に減衰させることで,探索初期では様々な関数へ,探索後期では距離の近い関数へ変更することを期待する. $ \epsilon_1 $ を1.0,減衰率を $ 0.99 $ としたとき,300世代目の $ \epsilon_300 $ は0.05程度になる.

この他に,区間積分差を用いた手法では,どちらの実験でも $ -1 $ から $ +1 $ までの範囲を積分し距離を計算する.個体の経験に基づく出力差を用いた手法では,どちらの実験でもミニバッチサイズは $ 24 $ とする.既存手法は現在選択されている関数以外の関数へランダムに変更されるため, $ \epsilon $ の値はプログラムに影響しない.

本実験を行うにあたって,すべての実験で共通するパラメータを以下の表に示す.

\begin{table}[h]
    \caption{使用するパラメータ}
    \centering
    \begin{tabular}{ll}
        \hline
        パラメータ & 値 \\
        \hline \hline
        個体数 & 480 \\
        最大世代数 & 300 \\
        変異の際,活性化関数の変更が選択される確率 & 50\% \\
        適応度測定の繰り返し回数 & 4 \\
        選択の際淘汰される確率 & 10\% \\
        選択の際エリート保存される確率 & 10\% \\
        \hline
    \end{tabular}
\end{table}

ひとつの世代には480体の個体が存在し,$ -2.0, -1.0, -0.5, +0.5, +1.0, +2.0$ のそれぞれの共有重みに対して4回の試行をし,劣悪な初期状態による適応度の低下を軽減する.そうして得られた480個の適応度のうち,上位10\%である48体のネットワーク構造は変更されることなく次世代の個体として保存され,下位10\%である48体の個体は淘汰され次世代には受け継がれない.上位90\%である432体のネットワークはWANNsに従い,ノードの挿入,シナプスの追加,活性化関数の変更のいずれかの操作を行い次世代の個体となる.このとき活性化関数の変更が変異として選択される確率は50\%である.この操作を300世代目まで行い,その時点で最も適応度の高い個体の出力を準最適解として記録する.

\subsection{実験結果}
