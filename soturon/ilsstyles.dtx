% \iffalse meta-comment
%% File ilsstyles.dtx
% \fi
%
% \CheckSum{0}
%
% \setcounter{StandardModuleDepth}{1}
% \StopEventually{}
%
% \iffalse
% \changes{v0.1--3}{1995/10/14}{first edition, 三期生: 美吉 明浩君により
%     \LaTeX のスタイルファイル"ils卒論.sty" として作成される。
%     その後数回の改良あり}
% \changes{v1.0a}{1999/12/19}{山口により p\LaTeX2e バージョンを作成}
% \changes{v1.1a}{2000/12/21}{修士論文スタイルを追加}
% \changes{v2.0a}{2004/2/27}{山口研究室に変更(ILS の名称は変えない)}
% \changes{v2.1a}{2004/11/11}{ilsfonts.sty を組み込みました}
% \changes{v2.1b}{2005/12/15}{表紙のフォント定義の方法を変更しました}
% \changes{v2.1c}{2013/9/24}{utf-8に日本語コードを変更しました}
% \fi
%
% \iffalse
%<soturon|shuron|driver|fonts|12pt>\NeedsTeXFormat{pLaTeX2e}
%<*driver>
\ProvidesFile{ilsclasses.dtx}
%</driver>
%<soturon>\ProvidesFile{ilssoturon.sty}
%<shuron>\ProvidesFile{ilsshuron.sty}
%<12pt>\ProvidesFile{ils12pt.sty}
%<fonts>\ProvidesFile{ilsfonts.sty}
%<soturon|shuron|driver|fonts>    [2005/12/15 v2.1b]
%<12pt>    [1999/12/19 v1.0a I.L.S. style file (size option)]
%<*driver>
\documentclass{jltxdoc}
\GetFileInfo{ilsstyles.dtx}
%\title{板倉研究室卒業論文用スタイル}
\title{山口研究室卒業論文用スタイル}
\author{山口 智}
%\date{作成日:2000年12月21日}
%\date{作成日:2004年11月11日}
\date{作成日:200l年12月15日}
\begin{document}
  \maketitle
  \tableofcontents
  \DocInput{\filename}
\end{document}
%</driver>
% \fi
%
% \section{オプションスイッチ}
% ここでは、後ほど使用するいくつかのコマンドやスイッチを定義しています。
%
%
% \begin{macro}{\if@HeaderWithUnderline}
% ページヘッダーの下に線を引くかどうかのスイッチです。
%    \begin{macrocode}
%<*soturon|shuron>
\newif\if@HeaderWithUnderline
  \@HeaderWithUnderlinefalse
%</soturon|shuron>
%    \end{macrocode}
% \end{macro}
%
%
% \begin{macro}{\if@ilschar}
% I.L.S フォントが定義されているかどうかのスイッチです。
% 工大マークなどが利用可能になります。
% ilschar.sty が必要です。
%    \begin{macrocode}
%<soturon|shuron>\newif\if@ilschar
%    \end{macrocode}
% \end{macro}
%
% \section{オプションの宣言}
%
% ここでは、クラスオプションの宣言を行なっています。
%
% \subsection{サイズオプション}
% 基準となるフォントの大きさを指定するオプションです。
%
% 板倉研の論文では 12pt がデフォルトです。現在のところ
% 11pt, 10pt がありません。
%    \begin{macrocode}
%<*soturon>
\if@compatibility
  \renewcommand{\@ptsize}{2}
\else
  \DeclareOption{12pt}{\renewcommand{\@ptsize}{2}}
\fi
\DeclareOption{11pt}{%
	\typeout{ポイントサイズ 11pt はありません。}
	\typeout{12pt に変更します。}
	\renewcommand{\@ptsize}{2}
}
\DeclareOption{10pt}{
	\typeout{ポイントサイズ 10pt はありません。}
	\typeout{12pt に変更します。}
	\renewcommand{\@ptsize}{2}
}
%</soturon>
%    \end{macrocode}
%
% \subsection{HeaderWithUnderline オプション}
% I.L.S ページスタイルでヘッダーの下に線を引く設定
% デフォルトは false です。
%    \begin{macrocode}
%<soturon|shuron>\DeclareOption{HeaderWithUnderline}{\@HeaderWithUnderlinetrue}
%    \end{macrocode}
%
% \subsection{ilschar オプション}
% I.L.S フォントの利用を有効にします。
%    \begin{macrocode}
%<*soturon|shuron>
\DeclareOption{ilschar}{%
  \input ilschar.sty%
  \@ilschartrue
}
%</soturon|shuron>
%    \end{macrocode}
%
% \subsection{オプションの実行}
% オプションの実行、およびサイズクラスのロードを行ないます。
%    \begin{macrocode}
%<*soturon|shuron>
\ExecuteOptions{12pt,oneside,onecolumn,final}
\ProcessOptions\relax
\input{ils1\@ptsize pt.sty}
%</soturon|shuron>
%    \end{macrocode}
%
% \subsection{卒論・修論の分類}
%    \begin{macrocode}
%<soturon> \def\@論文種類{卒業論文}
%<shuron> \def\@論文種類{修士論文}
%    \end{macrocode}
%
% \section{レイアウト}
% \subsection{用紙サイズの決定}
% A4 用紙サイズの設定です。
%    \begin{macrocode}
%<*soturon|shuron>
\setlength\paperheight {297mm}%
\setlength\paperwidth  {210mm}
%</soturon|shuron>
%    \end{macrocode}
%
% \subsection{段落の形}
%
% \begin{macro}{\lineskip}
% \begin{macro}{\normallineskip}
% これらの値は、行が近付き過ぎたときの\TeX の動作を制御します。
% 美吉版 卒論12pt.sty の定義を流用しています。
%    \begin{macrocode}
%<*12pt>
\setlength\lineskip{1pt}
\setlength\normallineskip{1pt}
%</12pt>
%    \end{macrocode}
% \end{macro}
% \end{macro}
%
% \begin{macro}{\baselinestretch}
% これは、|\baselineskip|の倍率を示すために使います。
% デフォルトでは、1.5 です。このコマンドが``empty''でない場合、
% |\baselineskip|の指定の\texttt{plus}や\texttt{minus}部分は無視される
% ことに注意してください。
%    \begin{macrocode}
%<12pt>\renewcommand{\baselinestretch}{1.5}
%    \end{macrocode}
% \end{macro}
%
% \begin{macro}{\parskip}
% \begin{macro}{\parindent}
% |\parskip|は段落間に挿入される、縦方向の追加スペースです。
% |\parindent|は段落の先頭の字下げ幅です。
% 美吉版 卒論12pt.sty の定義を流用しています。
%    \begin{macrocode}
%<*12pt>
\setlength\parskip{0pt}
\setlength\parindent{1zw}
%</12pt>
%    \end{macrocode}
% \end{macro}
% \end{macro}
%
%  \begin{macro}{\smallskipamount}
%  \begin{macro}{\medskipamount}
%  \begin{macro}{\bigskipamount}
% これら3つのパラメータの値は、\LaTeX{}カーネルの中で設定されています。
% これらはおそらく、サイズオプションの指定によって変えるべきです。
% しかし、\LaTeX~2.09や\LaTeXe{}の以前のリリースの両方との互換性を保つために、
% これらはまだ同じ値としています。
%    \begin{macrocode}
%<*12pt>
\setlength\smallskipamount{3\p@ \@plus 1\p@ \@minus 1\p@}
\setlength\medskipamount{6\p@ \@plus 2\p@ \@minus 2\p@}
\setlength\bigskipamount{12\p@ \@plus 4\p@ \@minus 4\p@}
%</12pt>
%    \end{macrocode}
%  \end{macro}
%  \end{macro}
%  \end{macro}
%
% \begin{macro}{\@lowpenalty}
% \begin{macro}{\@medpenalty}
% \begin{macro}{\@highpenalty}
% |\nopagebreak|と|\nolinebreak|コマンドは、これらのコマンドが置かれた場所に、
% ペナルティを起いて、分割を制御します。
% 置かれるペナルティは、コマンドの引数によって、
% |\@lowpenalty|, |\@medpenalty|, |\@highpenalty|のいずれかが使われます。
%    \begin{macrocode}
%<*soturon|shuron>
\@lowpenalty   51
\@medpenalty  151
\@highpenalty 301
%</soturon|shuron>
%    \end{macrocode}
% \end{macro}
% \end{macro}
% \end{macro}
%
% \subsection{ページレイアウト}
%
% \subsubsection{縦方向のスペース}
%
% \begin{macro}{\headheight}
% \begin{macro}{\headsep}
% \begin{macro}{\topskip}
% |\headheight|は、ヘッダが入るボックスの高さです。
% |\headsep|は、ヘッダの下端と本文領域との間の距離です。
% |\topskip|は、本文領域の上端と1行目のテキストのベースラインとの距離です。
%    \begin{macrocode}
%<*12pt>
\setlength\headheight{12pt} % ヘッダの高さ(4.2mm)
\setlength\headsep{45pt}    % ヘッダと本文の距離(15.75mm)
\setlength\topskip{-20pt}   % 上余白(-7mm)
%</12pt>
%    \end{macrocode}
% \end{macro}
% \end{macro}
% \end{macro}
%
% \begin{macro}{\footskip}
% |\footskip|は、本文領域の下端とフッタの下端との距離です。
% フッタのボックスの高さを示す、|\footheight|は削除されました。
%    \begin{macrocode}
%<12pt>\setlength\footskip{73pt} % (25.55mm)
%    \end{macrocode}
% \end{macro}
%
% \begin{macro}{\maxdepth}
% \TeX のプリミティブレジスタ|\maxdepth|は、|\topskip|と同じような
% 働きをします。|\@maxdepth|レジスタは、つねに|\maxdepth|のコピーでなくては
% いけません。これは|\begin{document}|の内部で設定されます。
% \TeX{}と\LaTeX~2.09では、|\maxdepth|は\texttt{4pt}に固定です。
% \LaTeXe{}では、|\maxdepth|$+$|\topskip|を基本サイズの1.5倍にしたいので、
% |\maxdepth|を|\topskip|の半分の値で設定します。
%    \begin{macrocode}
%<*12pt>
\if@compatibility
  \setlength\maxdepth{4\p@}
\else
  \setlength\maxdepth{.5\topskip}
\fi
%</12pt>
%    \end{macrocode}
% \end{macro}
%
% \subsubsection{本文領域}
% |\textheight|と|\textwidth|は、本文領域の通常の高さと幅を示します。
% 縦組でも横組でも、``高さ''は行数を、``幅''は字詰めを意味します。
% 後ほど、これらの長さに|\topskip|の値が加えられます。
%
% \begin{macro}{\textwidth}
% 基本組の字詰めです。
% 大きさの値は美吉版 卒論12pt.sty の定義を流用しています。
%
%    \begin{macrocode}
%<*12pt>
\setlength\textwidth{465pt}  % 本文幅(162.75mm)
\@settopoint\textwidth
%</12pt>
%    \end{macrocode}
% \end{macro}
%
% \begin{macro}{\textheight}
% 基本組の行数です。
% 大きさの値は美吉版 卒論12pt.sty の定義を流用しています。
%    \begin{macrocode}
%<*12pt>
\setlength\textheight{630pt}  % 本文高さ(220.5mm)
\@settopoint\textheight
%    \end{macrocode}
% 最後に、|\textheight|に|\topskip|の値を加えます。
%    \begin{macrocode}
\addtolength\textheight{\topskip}
\@settopoint\textheight
%</12pt>
%    \end{macrocode}
% \end{macro}
%
% \subsubsection{マージン}
%
% \begin{macro}{\topmargin}
% |\topmargin|は、``印字可能領域''---用紙の上端から1インチ内側---%
% の上端からヘッダ部分の上端までの距離です。
% 大きさの値は美吉版 卒論12pt.sty の定義を流用しています。
%
%    \begin{macrocode}
%<12pt> \setlength\topmargin{-20pt} % 上余白 (-7mm)
%    \end{macrocode}
% \end{macro}
%
% \begin{macro}{\marginparsep}
% \begin{macro}{\marginparpush}
% |\marginparsep|は、本文と傍注の間にあけるスペースの幅です。
% 本文の左(右)端と傍注の間になります。
% |\marginparpush|は、傍注と傍注との間のスペースの幅です。
%
% 大きさの値は美吉版 卒論12pt.sty の定義を流用しています。
%    \begin{macrocode}
%<*12pt>
\setlength\marginparsep{10pt} % (3.5mm)
\setlength\marginparpush{7pt} % (2.45mm)
%</12pt>
%    \end{macrocode}
% \end{macro}
% \end{macro}
%
% \begin{macro}{\oddsidemargin}
% \begin{macro}{\evensidemargin}
% \begin{macro}{\marginparwidth}
% 両サイドに書くことを考えていません。
%
% 大きさの値は美吉版 卒論12pt.sty の定義を流用しています。
%    \begin{macrocode}
%<*12pt>
\setlength\oddsidemargin{13pt}
\setlength\evensidemargin{0pt}
\setlength\marginparwidth{68pt}
%</12pt>
%    \end{macrocode}
% \end{macro}
% \end{macro}
% \end{macro}
%
% \subsection{脚注}
%
% \begin{macro}{\footnotesep}
% |\footnotesep|は、それぞれの脚注の先頭に置かれる``支柱''の高さです。
% このクラスでは、通常の|\footnotesize|の支柱と同じ長さですので、
% 脚注間に余計な空白は入りません。
%    \begin{macrocode}
%<12pt>\setlength\footnotesep{8.4pt}
%    \end{macrocode}
% \end{macro}
%
% \begin{macro}{\footins}
% |\skip\footins|は、本文の最終行と最初の脚注との間の距離です。
%    \begin{macrocode}
%<12pt>\setlength{\skip\footins}{10.8pt \@plus 4pt minus 2pt}
%    \end{macrocode}
% \end{macro}
%
% \subsection{ilssitagakiスタイル}
%
% \begin{macro}{\ps@ilssitagaki}
% \pstyle{ilssitagaki}は添削を受けるための
% 下書き用ページスタイルです。ヘッダーの右側にサブセクション
% の代わりに日付を入れます。
%    \begin{macrocode}
%<*soturon|shuron>
\newif\if@ilssitagaki \@ilssitagakifalse % 下書きである時に true
\def\ps@ilssitagaki{\let\ps@jpl@in\ps@psplain%
  \let\@evenhead\@empty%
  \if@HeaderWithUnderline
    \def\@oddhead{\hskip -25pt%          <- 奇数頁ヘッダの定義始まり
    \underline{%
      \hbox to 1.065\textwidth{%
      \small \bf \leftmark \hfill%       <- 左上頁に出力したい内容
       \西暦\today\hbox{}}}}%             <- 右上頁に出力したい内容
  \else
    \def\@oddhead{\hbox{}\small%        <- 奇数頁ヘッダの定義始まり
      \bf \leftmark \hfill %             <- 左上頁に出力したい内容
      \西暦\today\hbox{}}%               <- 右上頁に出力したい内容
  \fi
  \def\sectionmark##1{\markboth{%      <- セクションを表示したい設定
    \ifnum \c@secnumdepth >\z@ %
      {\bf\thesection}\hskip 1em\relax %
    \fi %
    ##1}{}}%
  \def\subsectionmark##1{\markright{%     <- サブセクションを表示したい時
%                                           小節見出しを柱に入れる命令。
    \ifnum \c@secnumdepth >\z@ %   <- もし secnumdepth カウンタの値>0なら
      \hfill{\bf\thesubsection}%      <-小節番号表示
      \hskip 1em\relax%
    \fi%
    ##1}}% <- 小節の名前表示
%
  \let\@evenfoot\@empty%
  \def\@oddfoot{
    \hbox to 0.5\textwidth{\hfil --~\lower0.2ex\hbox{\thepage} --}%
%    \hbox to 0.5\textwidth{\hfil {%
    {\hfil {%
    \raisebox{-0.7zh}{%
      \lower0.1zh\hbox{\Koudai\,}
      \shortstack[l]{%
        {\small 情報工学科山口研究室}\\%
        \raise0.0zh\hbox{\small 千葉工業大学\@論文種類}}}}}}%
  \let\@mkboth\@gobbletwo
  \@ilssitagakitrue
}%                                           <- 定義ここまで
%</soturon|shuron>
%    \end{macrocode}
% \end{macro}
%
% \subsection{ilsheadingsスタイル(卒論用)}
% \begin{macro}{\ps@ilsheadings}
% \pstyle{ilsheadings}は山口研卒業論文用のページスタイルです。
% これも美吉君の``ils卒論.sty" を参考に作成しました。
%    \begin{macrocode}
%<*soturon>
\def\ps@ilsheadings{\let\ps@jpl@in\ps@psplain%
  \let\@evenhead\@empty%
  \if@HeaderWithUnderline
    \def\@oddhead{\hskip -25pt%          <- 奇数頁ヘッダの定義始まり
    \underline{%
      \hbox to 1.065\textwidth{%
      \small \bf \hfil \leftmark%       <- 左上頁に出力したい内容
       \rightmark\hbox{}}}}%             <- 右上頁に出力したい内容
  \else
    \def\@oddhead{\hbox{}\small%        <- 奇数頁ヘッダの定義始まり
      \bf \hfil \leftmark %             <- 左上頁に出力したい内容
      \rightmark\hbox{}}%               <- 右上頁に出力したい内容
  \fi
  \def\sectionmark##1{\markboth{%      <- セクションを表示したい設定
    \ifnum \c@secnumdepth >\z@ %
      {\bf\thesection}\hskip 1em\relax %
    \fi %
    ##1}{}}%
  \def\subsectionmark##1{\markright{%     <- サブセクションを表示したい時
%                                           小節見出しを柱に入れる命令。
    \ifnum \c@secnumdepth >\z@ %   <- もし secnumdepth カウンタの値>0なら
      \hfill{\bf\thesubsection}%      <-小節番号表示
      \hskip 1em\relax%
    \fi%
    ##1}}% <- 小節の名前表示
%
  \let\@evenfoot\@empty%
  \def\@oddfoot{
    \hbox to 0.5\textwidth{\hfil --~\lower0.2ex\hbox{\thepage} --}%
%    \hbox to 0.5\textwidth{\hfil {%
    {\hfil {%
    \raisebox{-0.7zh}{%
      \lower0.1zh\hbox{\Koudai\,}
      \shortstack[l]{%
        {\small 情報工学科山口研究室}\\%
        \raise0.0zh\hbox{\small 千葉工業大学\@論文種類}}}}}}%
  \let\@mkboth\@gobbletwo
}%                                           <- ILS卒論header 定義ここまで
%</soturon>
%    \end{macrocode}
% \end{macro}
%
% \subsection{ilsheadingsスタイル(修論用)}
%
% \begin{macro}{\ps@ilsheadings}
% \pstyle{ilsheadings}は板倉研卒業論文用のページスタイルです。
% これも美吉君の``ils卒論.sty" を参考に作成しました。板倉研の
% 修論は Chapter を章立てとして使っているので、それに合わせてあります。
% jreport.cls を使う際の Chapter の変更も行っています。
%    \begin{macrocode}
%<*shuron>
\def\ps@ilsheadings{\let\ps@jpl@in\ps@psplain%
  \let\@evenhead\@empty%
  \if@HeaderWithUnderline
    \def\@oddhead{\hskip -25pt%          <- 奇数頁ヘッダの定義始まり
    \underline{%
      \hbox to 1.065\textwidth{%
      \small \bf \hfil \leftmark%       <- 左上頁に出力したい内容
       \rightmark\hbox{}}}}%             <- 右上頁に出力したい内容
  \else
    \def\@oddhead{\hbox{}\small%        <- 奇数頁ヘッダの定義始まり
      \bf \hfil \leftmark %             <- 左上頁に出力したい内容
      \rightmark\hbox{}}%               <- 右上頁に出力したい内容
  \fi
  \def\chaptermark##1{\markboth{%      <- チャプターを表示したい設定
    \ifnum \c@secnumdepth >\z@ %
      {\bf\thechapter}\hskip 1em\relax %
    \fi %
    ##1}{}}%
  \def\sectionmark##1{\markright{%     <- セクションを表示したい時
%                                           小節見出しを柱に入れる命令。
    \ifnum \c@secnumdepth >\z@ %   <- もし secnumdepth カウンタの値>0なら
      \hfill{\bf\thesection}%      <-小節番号表示
      \hskip 1em\relax%
    \fi%
    ##1}}% <- 小節の名前表示
%
  \let\@evenfoot\@empty%
  \def\@oddfoot{
    \hbox to 0.5\textwidth{\hfil --~\lower0.2ex\hbox{\thepage} --}%
%    \hbox to 0.5\textwidth{\hfil {%
    {\hfil {%
    \raisebox{-0.7zh}{%
      \lower0.1zh\hbox{\Koudai\,}
      \shortstack[l]{%
        {\small 情報工学科山口研究室}\\%
        \raise0.0zh\hbox{\small 千葉工業大学\@論文種類}}}}}}%
  \let\@mkboth\@gobbletwo
}%                                           <- ILS卒論header 定義ここまで
%%%% jreport.cls の chapter の変更
\renewcommand{\chapter}{%
  \if@openright\cleardoublepage\else\clearpage\fi
  %\thispagestyle{jpl@in}% <-- この行をコメントアウト
  \global\@topnum\z@
  \@afterindentfalse
  \secdef\@chapter\@schapter}


\def\@chapter[#1]#2{%
  \ifnum \c@secnumdepth >\m@ne
    \refstepcounter{chapter}%
    \typeout{\@chapapp\space\thechapter\space\@chappos}%
    \addcontentsline{toc}{chapter}%
      {\protect\numberline{\@chapapp\thechapter\@chappos}#1}%
  \else
    \addcontentsline{toc}{chapter}{#1}%
  \fi
  \chaptermark{#1}%
  \addtocontents{lof}{\protect\addvspace{10\p@}}%
  \addtocontents{lot}{\protect\addvspace{10\p@}}%
  \@makechapterhead{#2}\@afterheading
	\thispagestyle{empty}%                    <-- ページスタイルを変更
	\clearpage%                             <-- 改ページ
}

\def\@schapter#1{%
  \@makeschapterhead{#1}\@afterheading
	\if@istoc
	\else
	\thispagestyle{empty}%
	\clearpage% <--- 番号なしのときも同じ処理 (目次に限って改ページなし)
	\fi
}
%</shuron>
%    \end{macrocode}
% \end{macro}
%
% \section{卒論表紙}
%
% 板倉研卒論用タイトルページを出力します。
% 美吉版の書き方とは大きく違っています(出力される形式はほぼ同じ)。
% 互換性がありません。
%
% \subsection{データ入力用変数}
% この項目はこれまで \verb|\token| で記述されていました。
% 本スタイルでは変数として記述します。このため定義の仕方に
% 変更があります。たとえば\\
% \verb|\年度={1999年度} は \年度{1999年度}|\\
% のように = をつけないでください。
%
% 別の変更として論文題目や研究者を tabular 環境で書くことにしました。
% 論文の題目が長いときは\\
% \verb|\論文題目{論文題目前半\\論文題目後半}|\\
% といったように\verb|\\|で区切ってつなげてください。
% また、研究者の書き方は\\
% \verb|\研究者{学生番号1& 氏名1\\学生番号2&氏名2}|\\
% となります。題目、研究者とも 3 行くらいならきれいに書けます。
%
%    \begin{macrocode}
%<*soturon>
\def\年度#1{\def\@年度{#1}}
\def\提出日#1{\def\@提出日{#1}}
\def\研究者#1{\def\@研究者{#1}}
\def\論文題目#1{\def\@論文題目{#1}}
%</soturon>
% 修士論文の場合は、英文題目や要旨の記述の関係上学生番号と氏名を
% 分けて書きます。
%<*shuron>
% 研究者・論文題目・和文専攻・英文専攻は使いません
\def\研究者#1{\typeout{`\\研究者'は使えません}}
\def\論文題目#1{\typeout{論文題目は使えません}}
\def\和文専攻#1{\typeout{和文専攻は情報工学で固定です}}
\def\英文専攻#1{\typeout{英文専攻は Computer Science で固定です}}
%以下の項目は互換性のための定義です。
\def\和文論文題目#1{\和文題目{#1}}
\def\英文論文題目#1{\英文題目{#1}}
\def\和文論文題目前半#1{\def\@和文題目前半{#1}}
\def\和文論文題目後半#1{\和文題目{\@和文題目前半\\#1}}
\def\和文鍵言葉1#1{\def\@和文鍵言葉1{#1}\和文キーワード{#1}}
\def\和文鍵言葉2#1{\def\@和文鍵言葉2{#1}\和文キーワード{\@和文鍵言葉1、#1}}
\def\和文鍵言葉3#1{\def\@和文鍵言葉3{#1}\和文キーワード{\@和文鍵言葉1、\@和文鍵言葉2、#1}}
\def\和文鍵言葉4#1{\def\@和文鍵言葉4{#1}\和文キーワード{\@和文鍵言葉1、\@和文鍵言葉2、\@和文鍵言葉3、#1}}
\def\和文鍵言葉5#1{\def\@和文鍵言葉5{#1}\和文キーワード{\@和文鍵言葉1、\@和文鍵言葉2、\@和文鍵言葉3、\@和文鍵言葉4、#1}}
\def\和文鍵言葉6#1{\def\@和文鍵言葉6{#1}\和文キーワード{\@和文鍵言葉1、\@和文鍵言葉2、\@和文鍵言葉3、\@和文鍵言葉4、\@和文鍵言葉5、#1}}
\def\英文鍵言葉1#1{\def\@英文鍵言葉1{#1}\英文キーワード{#1}}
\def\英文鍵言葉2#1{\def\@英文鍵言葉2{#1}\英文キーワード{\@英文鍵言葉1, #1}}
\def\英文鍵言葉3#1{\def\@英文鍵言葉3{#1}\英文キーワード{\@英文鍵言葉1, \@英文鍵言葉2, #1}}
\def\英文鍵言葉4#1{\def\@英文鍵言葉4{#1}\英文キーワード{\@英文鍵言葉1, \@英文鍵言葉2, \@英文鍵言葉3, #1}}
\def\英文鍵言葉5#1{\def\@英文鍵言葉5{#1}\英文キーワード{\@英文鍵言葉1, \@英文鍵言葉2, \@英文鍵言葉3, \@英文鍵言葉4, #1}}
\def\英文鍵言葉6#1{\def\@英文鍵言葉6{#1}\英文キーワード{\@英文鍵言葉1, \@英文鍵言葉2, \@英文鍵言葉3, \@英文鍵言葉4, \@英文鍵言葉5, #1}}
%%
\def\年度#1{\def\@年度{#1}}
\def\提出日#1{\def\@提出日{#1}}
\def\学生番号#1{\def\@学生番号{#1}}
\def\和文氏名#1{\def\@和文氏名{#1}}
\def\和文題目#1{\def\@和文題目{#1}}
\def\和文キーワード#1{\def\@和文キーワード{#1}}
\def\英文氏名#1{\def\@英文氏名{#1}}
\def\英文題目#1{\def\@英文題目{#1}}
\def\英文キーワード#1{\def\@英文キーワード{#1}}
\def\@論文題目{\@和文題目}
\def\@研究者{\@学生番号 & \@和文氏名}
\newtoks\和文論文要旨
\newtoks\英文論文要旨
%\def\和文論文要旨#1{\def\@和文論文要旨{#1}}
%\def\英文論文要旨#1{\def\@英文論文要旨{#1}}
%</shuron>
%    \end{macrocode} 
% \subsection{表紙の定義}
% 卒論・修論用表紙を出力するマクロ
%
%    \begin{macrocode}
%<*12pt>
% 表紙のためのフォント定義
  \newfont{\egtrm}{cmr8}
  \newfont{\tenrm}{cmr10}
  \newfont{\twlrm}{cmr10 scaled \magstep 1}
  \newfont{\frtnrm}{cmr10 scaled \magstep 2}
  \newfont{\svtnrm}{cmr10 scaled \magstep 3}
  \newfont{\twtyrm}{cmr10 scaled \magstep 4}
  \newfont{\twfvrm}{cmr10 scaled \magstep 5}
  \newfont{\twtybf}{cmbx10 scaled \magstep 4}
  \newfont{\twfvbf}{cmbx10 scaled \magstep 5}
  \newfont{\svtngt}{goth10 scaled \magstep 3}
  \newfont{\twtygt}{goth10 scaled \magstep 4}
  \newfont{\twfvgt}{goth10 scaled \magstep 5}
  \newfont{\tenmin}{min10}
  \newfont{\twlmin}{min10 scaled \magstep 1}
  \newfont{\frtnmin}{min10 scaled \magstep 2}
  \newfont{\svtnmin}{min10 scaled \magstep 3}
  \newfont{\twtymin}{min10 scaled \magstep 4}
  \newfont{\twfvmin}{min10 scaled \magstep 5}
  \newfont{\twfrils}{ils scaled \magstep 4}
  \newfont{\twfvils}{ils scaled \magstep 5}
  \newfont{\thtyils}{ils scaled 2986}
%</12pt>
%    \end{macrocode}
%\begin{macro}{\卒論表紙}
%    \begin{macrocode}
%<*soturon>
\def\卒論表紙{{
\thispagestyle{empty}
\unitlength=1pt
\noindent
\begin{picture}(465,600)(0,-604)
%%%%\put(0,-25){\makebox(465,25){\twfvrm\twfvmin \@年度}}
%%%%\put(0,-65){\makebox(465,25){\twfvmin \@論文種類}}
\put(0,-25){\makebox(465,25){\Huge \@年度}}
\put(0,-65){\makebox(465,25){\Huge \@論文種類}}
\put(0,-140){\makebox(465,50){\LARGE\Koudai}}
%%%%\put(0,-175){\makebox(465,20){\twtymin 論文題目}}
\put(0,-175){\makebox(465,20){\LARGE 論文題目}}
%%%%\renewcommand{\arraystretch}{1.2}
\renewcommand{\arraystretch}{0.6}
%%%\put(0,-290){\makebox(465,100)[t]{\twfvgt\twfvbf
\put(0,-290){\makebox(465,100)[t]{\textbf{\Huge
	\begin{tabular}{c} 
	\@論文題目
	\end{tabular}
	}}}
%%%%\put(0,-340){\makebox(465,20){\twtymin 研究者}}
\put(0,-340){\makebox(465,20){\LARGE 研究者}}
%%%\renewcommand{\arraystretch}{1.4}
\renewcommand{\arraystretch}{1.0}
%%%%\put(0,-440){\makebox(465,100)[t]{\twtygt\twtybf
\put(0,-440){\makebox(465,100)[t]{\textbf{\LARGE
	\begin{tabular}{cl}
	\@研究者
	\end{tabular}
	}}}
%%%\renewcommand{\arraystretch}{1.4}
\renewcommand{\arraystretch}{1.0}
%%%\put(0,-485){\makebox(465,20){\twtymin 指導教員}}
\put(0,-485){\makebox(465,20){\LARGE 指導教員}}
%%%\put(0,-565){\makebox(465,75)[t]{\twtygt\twtybf
\put(0,-565){\makebox(465,75)[t]{\textbf{\LARGE
	\begin{tabular}{ll} 
	%板倉 秀清&教授\\
	山口 智 &准教授
	\end{tabular}}
	}}
\if@ilssitagaki
%%%\put(0,-600){\makebox(465,20){\frtnrm\frtnmin 添削用原稿提出日 \西暦\today}}
\put(0,-600){\makebox(465,20){\large 添削用原稿提出日 \西暦\today}}
\else
%%%\put(0,-600){\makebox(465,20){\frtnrm\frtnmin \@提出日}}
\put(0,-600){\makebox(465,20){\large \@提出日}}
\fi
\end{picture}
\setcounter{page}\@ne
\clearpage
}}
%</soturon>
%    \end{macrocode}
% \end{macro}
%
%\begin{macro}{\修論表紙}
%    \begin{macrocode}
%<*shuron>
\def\修論表紙{{
\thispagestyle{empty}
\unitlength=1pt
\noindent
\begin{picture}(465,600)(0,-604)
\put(0,-25){\makebox(465,25){\twfvrm\twfvmin \@年度}}
\put(0,-65){\makebox(465,25){\twfvmin \@論文種類}}
\put(0,-140){\makebox(465,50){\LARGE\Koudai}}
\put(0,-175){\makebox(465,20){\twtymin 論文題目}}
\renewcommand{\arraystretch}{1.2}
\put(0,-290){\makebox(465,100)[t]{\twfvgt\twfvbf
	\begin{tabular}{c} 
	\@論文題目
	\end{tabular}
	}}
\put(0,-340){\makebox(465,30){\twtymin
% \begin{tabular}{c} 千葉工業大学大学院工学研究科\\
\begin{tabular}{c} 千葉工業大学大学院情報科学研究科\\
情報工学専攻博士前期課程\end{tabular}}}
\renewcommand{\arraystretch}{1.4}
\put(0,-450){\makebox(465,100)[t]{\twtygt\twtybf
	\begin{tabular}{cl}
	\@研究者
	\end{tabular}
	}}
\renewcommand{\arraystretch}{1.4}
\put(0,-485){\makebox(465,20){\twtymin 指導教員}}
\put(0,-565){\makebox(465,75)[t]{\twtygt\twtybf
	\begin{tabular}{ll} 
	%板倉 秀清&教授\\
	山口 智 &准教授
	\end{tabular}
	}}
\if@ilssitagaki
\put(0,-600){\makebox(465,20){\frtnrm\frtnmin 添削用原稿提出日 \西暦\today}}
\else
\put(0,-600){\makebox(465,20){\frtnrm\frtnmin \@提出日}}
\fi
\end{picture}
\setcounter{page}\@ne
\clearpage
}}
%</shuron>
%    \end{macrocode}
% \end{macro}
%
% \section{目次の修正}
% \begin{macro}{\tableofcontents}
%     \begin{macrocode}
%<*shuron>
\newif\if@istoc
\renewcommand{\tableofcontents}{%
	\pagenumbering{roman}%     <- 目次はローマ数字でページ振り
	\addcontentsline{toc}{chapter}{目 次}%  -> 目次に目次も入れたい!
	\@istoctrue% <-- 目次の場合だけchapterに改ページを入れない
	\markboth{目次}{}
  \if@twocolumn\@restonecoltrue\onecolumn
  \else\@restonecolfalse\fi
  \chapter*{\contentsname
    \@mkboth{\contentsname}{\contentsname}%
  }\@starttoc{toc}%
  \if@restonecol\twocolumn\fi
	\clearpage
	\pagenumbering{arabic}%   <- 本文はアラビア数字でページ振り
	\@istocfalse% <-- これ以降は目次じゃない
}
%</shuron>
%    \end{macrocode}
% \end{macro}
%
%% \section{謝辞の修正}
% \begin{macro}{謝辞}
%     \begin{macrocode}
%<*soturon>
\def\謝辞{
\section*{謝辞} % 番号なしのセクション
\addcontentsline{toc}{section}{謝辞}
%\refname\markboth{謝辞}{}
\markboth{謝辞}{}
}
%</soturon>
%<*shuron>
\def\謝辞{
\chapter*{謝辞} % 番号なしのセクション
\markboth{謝辞}{}
\addcontentsline{toc}{chapter}{謝辞}
}
%</shuron>
%    \end{macrocode}
% \end{macro}
% \section{参考文献の再定義}
%
% \begin{macro}{\bibindent}
% オープンスタイルの参考文献で使うインデント幅です。
%    \begin{macrocode}
%<*soturon|shuron>
\newdimen\bibindent
\setlength\bibindent{1.5em}
%</soturon|shuron>
%    \end{macrocode}
% \end{macro}
%
% \begin{macro}{\newblock}
% |\newblock|のデフォルト定義は、小さなスペースを生成します。
%    \begin{macrocode}
%<*soturon|shuron>
\renewcommand{\newblock}{\hskip .11em\@plus.33em\@minus.07em}
%</soturon|shuron>
%    \end{macrocode}
% \end{macro}
%
% \begin{environment}{thebibliography}
% 参考文献や関連図書のリストを作成します。
%    \begin{macrocode}
%<*soturon>
\renewenvironment{thebibliography}[1]
{\section*{\refname\markboth{参考文献}{}}%
   \list{\@biblabel{\@arabic\c@enumiv}}%
        {\settowidth\labelwidth{\@biblabel{#1}}%
         \leftmargin\labelwidth
         \advance\leftmargin\labelsep
         \@openbib@code
         \usecounter{enumiv}%
         \let\p@enumiv\@empty
         \renewcommand\theenumiv{\@arabic\c@enumiv}}%
   \sloppy
%    \end{macrocode}
%    \begin{macrocode}
   \clubpenalty4000
   \@clubpenalty\clubpenalty
   \widowpenalty4000%
   \sfcode`\.\@m}
  {\def\@noitemerr
    {\@latex@warning{Empty `thebibliography' environment}}%
   \endlist}
%</soturon>
%<*shuron> 
\renewcommand{\bibname}{参考文献}
\renewenvironment{thebibliography}[1]
{%\markboth{参考文献}{}%
\chapter*{\bibname\@mkboth{\bibname}{\bibname}}%
	\clearpage\markboth{参考文献}{}% <-- ヘッダの変更
   \list{\@biblabel{\@arabic\c@enumiv}}%
        {\settowidth\labelwidth{\@biblabel{#1}}%
         \leftmargin\labelwidth
         \advance\leftmargin\labelsep
         \@openbib@code
         \usecounter{enumiv}%
         \let\p@enumiv\@empty
         \renewcommand\theenumiv{\@arabic\c@enumiv}}%
   \sloppy
   \clubpenalty4000
   \@clubpenalty\clubpenalty
   \widowpenalty4000%
   \sfcode`\.\@m}

%</shuron>
%    \end{macrocode}
% \end{environment}
%
% \begin{macro}{\@openbib@code}
% |\@openbib@code|のデフォルト定義は何もしません。
% この定義は、\Lopt{openbib}オプションによって変更されます。
%    \begin{macrocode}
%<soturon|shuron>\let\@openbib@code\@empty
%    \end{macrocode}
% \end{macro}
%
% \begin{macro}{\@biblabel}
%    The label for a |\bibitem[...]| command is produced by this
%    macro. The default from \file{latex.dtx} is used.
%    \begin{macrocode}
% \renewcommand*{\@biblabel}[1]{[#1]\hfill}
%    \end{macrocode}
% \end{macro}
%
% \begin{macro}{\@cite}
%    The output of the |\cite| command is produced by this macro. The
%    default from \file{latex.dtx} is used.
%    \begin{macrocode}
% \renewcommand*{\@cite}[1]{[#1]}
%    \end{macrocode}
% \end{macro}
%
% \section{変数の再定義}
% ilsheadings ページスタイルを有効にします
%    \begin{macrocode}
%<soturon|shuron>\ps@ilsheadings
%    \end{macrocode}
%
% \section{論文要旨}
% 修士論文のための和文・英文要旨の定義
% \subsection{長さを設定する box の定義など}
%
%    \begin{macrocode}
%<*shuron>
%
% 位置の最終調整用変数
%
\newdimen\HOFFSET \HOFFSET=0mm
\newdimen\VOFFSET \VOFFSET=0mm
%
%%% オフセット %%%%%%%
\def\OFFSET{
  \advance\hoffset by \HOFFSET
  \advance\voffset by \VOFFSET
}
% working variables
%
\newbox\dummybox
\newdimen\X
\newdimen\Y
\newdimen\XXA
\newdimen\YYA
%
% memory for positions of rules
%
\newdimen\XM \XM=4mm% X-direction margin
\newdimen\YM \YM=2mm% Y-direction margin
%
\newdimen\htbackskip
\newdimen\dpbackskip
\long\def\XY#1#2#3{
  \setbox\dummybox=\vbox{
    \vskip#2
    \hbox to \hsize{%
      \hskip#1
      \vbox{\advance\hsize by -#1#3}\hfil
    }
  }
  \htbackskip=-\ht\dummybox
  \dpbackskip=-\dp\dummybox
%
  \box\dummybox\nointerlineskip
  \vskip\htbackskip
  \vskip\dpbackskip
}
%
% 絶対位置 (#1, #2) (#3, #4)で定まるボックスに
% #5 を入れて上下左右でセンタリングする。
%
\long\def\XYBC#1#2#3#4#5{
  \X=#3 \Y=#4 \advance\X by -#1 \advance\Y by -#2
  \XY{#1}{#2}{\CBOX{\X}{\Y}{#5}}
}
\long\def\XYBP#1#2#3#4#5{
  \X=#3 \Y=#4 \advance\X by -#1 \advance\Y by -#2
  \XY{#1}{#2}{\PBOX{\X}{\Y}{#5}}
}
\long\def\PBOX#1#2#3{
   \X=#1 \Y=#2 \advance\X by -\XM \XXA=\X \advance\X by -\XM
               \advance\Y by -\YM \YYA=\Y \advance\Y by -\YM
   \vbox to \YYA{\vskip\YM
   \hbox to \XXA{\hskip\XM\vbox to \Y{\hsize=\X\linewidth=\X \parbox{\hsize}{\parindent=1zw #3}\vfil}%
   \hfil}
}}

%
% BOX を描く。縦横にセンタリングされる.#1, #2引数は長さで与える
% 箱の横の長さ、縦の長さ、中身
%
\long\def\CBOX#1#2#3{{
  \hsize=#1\linewidth=\hsize
  \setbox\dummybox\hbox{#3}
  \vbox to #2{
    \vfil
    \ifdim#1 < \wd\dummybox
      \X=#1
      \advance\X by -4mm
      \hbox to \hsize{\hfil\vbox{\hsize=\X\linewidth=\hsize #3}\hfil}
    \else
      \hbox to \hsize{\hfil #3\hfil}
    \fi
    \vfil
  }
}}
%
%vertival rule with 0.7pt width
% 縦罫線 0.7pt 幅 始点座標 (#1, #2) 終点Y座標 #3
%
\newdimen\rulelength
\def\Vrule#1#2#3{
   \rulelength=#3
   \advance\rulelength by -#2
   \XY{#1}{#2}{\vrule width0.7pt height\rulelength depth0pt}
}
%
%
%horizontal rule with 0.7pt width
% 横罫線 0.7pt 幅 始点座標 (#1, #2) 終点X座標 #3
%
\def\Hrule#1#2#3{
   \rulelength=#3
   \advance\rulelength by -#1
   \XY{#1}{#2}{\vrule width\rulelength height0.7pt depth0pt}
}
%
%vertival rule with 1.4pt width
% 縦罫線 1.4pt 幅 始点座標 (#1, #2) 終点Y座標 #3
%
\def\Vrulethick#1#2#3{
   \rulelength=#3
   \advance\rulelength by -#2
   \XY{#1}{#2}{\vrule width1.4pt height\rulelength depth0pt}
}
%horizontal rule with 1.4pt width
% 横罫線 1.4pt 幅 始点座標 (#1, #2) 終点X座標 #3
%
\def\Hrulethick#1#2#3{
   \rulelength=#3
   \advance\rulelength by -#1
   \XY{#1}{#2}{\vrule width\rulelength height1.4pt depth0pt}
}
%</shuron>
%    \end{macrocode}
%
% \subsection{論文要旨}
% \begin{macro}{\論文要旨}
%    \begin{macrocode}
%<*shuron>
\newif\if和文\和文true
\def\和文要旨{\和文true  \論文要旨}
\def\英文要旨{\和文false \論文要旨}
\def\論文要旨{{
  \hoffset=0mm \voffset=0mm
  \advance\hsize by -\hoffset
  \advance\vsize by -\voffset
  \OFFSET
  \thispagestyle{empty}
  \headsep\z@ \headheight\z@ \oddsidemargin 2.5mm
  \topmargin -1.4mm
  \hsize=170mm \vsize=280mm
  \parindent\z@ \parskip\z@
  \linewidth=\hsize \baselineskip = 4.0mm plus1mm minus0.5mm
  \rightskip = 0pt plus 120mm \lineskip\z@ \topskip\z@
  \def\baselinestretch{1}\large\normalsize
%
  \Hrulethick{2.5mm}{23.5mm}{164.5mm} \Hrulethick{  2.5mm}{243.5mm}{164.5mm}
  \Vrulethick{2.5mm}{23.5mm}{243.5mm} \Vrulethick{164.5mm}{ 23.5mm}{244mm}

  \Hrule{2.5mm}{33.5mm}{164.5mm} \Hrule{2.5mm}{48.5mm}{164.5mm}
  \Hrule{2.5mm}{78.5mm}{164.5mm} \Hrule{2.5mm}{88.5mm}{164.5mm}

  \Vrule{52.5mm}{23.5mm}{48.5mm} \Vrule{92.5mm}{23.5mm}{48.5mm}

  \XYBC{2.5mm}{-9mm}{164.5mm}{10mm}{\kern-1.7zw\raisebox{-6zh}{ \Huge \Koudai}}
          \if和文
    \XYBC{2.5mm}{ 11.0mm}{164.5mm}{ 23.5mm}{\twfvmin 修士論文要旨}

	%%% 研究科名の変更にともなう変更
    %%%% \XYBC{2.5mm}{246.5mm}{164.5mm}{252.5mm}{\frtnmin 千葉工業大学大学院工学研究科}
    \XYBC{2.5mm}{246.5mm}{164.5mm}{252.5mm}{\frtnmin 千葉工業大学大学院情報科学研究科}
    \XYBC{2.5mm}{252.5mm}{164.5mm}{258.5mm}{\tenmin \tenrm ◎要旨は600字程度}

    \XYBC{ 2.5mm}{23.5mm}{ 52.5mm}{33.5mm}{\svtnmin 専 攻}
	%%% 研究科名の変更にともなう変更
    %%% \XYBC{ 2.5mm}{33.5mm}{ 52.5mm}{48.5mm}{\svtnmin 情報工学}
    \XYBC{ 2.5mm}{33.5mm}{ 52.5mm}{48.5mm}{\svtnmin 情報科学}
    \XYBC{52.5mm}{23.5mm}{ 92.5mm}{33.5mm}{\svtnmin 学生番号}
    \XYBC{52.5mm}{33.5mm}{ 92.5mm}{48.5mm}{\svtnrm\@学生番号}
    \XYBC{92.5mm}{23.5mm}{164.5mm}{33.5mm}{\svtnmin 氏  名}
    \XYBC{92.5mm}{33.5mm}{164.5mm}{48.5mm}{\svtnmin\@和文氏名}

    \XYBC{2.5mm}{48.5mm}{ 32.5mm}{58.5mm}{\svtnmin 論文題目}
    \XYBC{2.5mm}{58.5mm}{164.5mm}{78.5mm}{\svtnmin \svtnrm \begin{tabular}{c}\@論文題目\end{tabular}}

    \XYBC{ 2.5mm}{78.5mm}{ 32.5mm}{88.5mm}{\frtnmin キーワード}
    \XYBC{32.5mm}{78.5mm}{164.5mm}{88.5mm}{\tenmin\tenrm\baselineskip=1.5mm%
	\begin{tabular}{p{127.5mm}}\@和文キーワード\end{tabular}
    \baselineskip = 4.0mm plus1mm minus0.5mm}

    \XYBC{ 2.5mm}{ 88.5mm}{ 32.5mm}{100.5mm}{\svtnmin 論文要旨}
    \XYBP{12.5mm}{100.5mm}{154.5mm}{243.5mm}{\twlmin \the\和文論文要旨}
  \else
    \XYBC{2.5mm}{ 11.0mm}{164.5mm}{ 23.5mm}{\twtyrm Summary of Master's Thesis}

	%%% 研究科名の変更にともなう変更
    %%% \XYBC{2.5mm}{246.5mm}{164.5mm}{252.5mm}{\twlrm Graduate School of Engineering,\ Chiba Institute of Technology}
    \XYBC{2.5mm}{246.5mm}{164.5mm}{252.5mm}{\twlrm Graduate School of Information and Computer Science,\ Chiba Institute of Technology}
    \XYBC{2.5mm}{252.5mm}{164.5mm}{258.5mm}{\egtrm The contents of English summary should be writen in about 200 words.}

    \XYBC{ 2.5mm}{23.5mm}{ 52.5mm}{33.5mm}{\svtnrm Course}
	%%% 研究科名の変更にともなう変更
    %%% \XYBC{ 2.5mm}{33.5mm}{ 52.5mm}{48.5mm}{\frtnrm Computer Science}
    \XYBC{ 2.5mm}{33.5mm}{ 52.5mm}{48.5mm}{\frtnrm Information and Computer Science}
    \XYBC{52.5mm}{23.5mm}{ 92.5mm}{33.5mm}{\frtnrm Student No.}
    \XYBC{52.5mm}{33.5mm}{ 92.5mm}{48.5mm}{\frtnrm\@学生番号}
    \XYBC{92.5mm}{23.5mm}{164.5mm}{33.5mm}{\frtnrm SURNAME,\ Firstname}
    \XYBC{92.5mm}{33.5mm}{164.5mm}{48.5mm}{\frtnrm\@英文氏名}

    \XYBC{2.5mm}{48.5mm}{ 20.5mm}{58.5mm}{\svtnrm Title}
    \XYBC{2.5mm}{58.5mm}{164.5mm}{78.5mm}{\svtnmin \svtnrm \begin{tabular}{c}\@英文題目\end{tabular}}

    \XYBC{ 2.5mm}{78.5mm}{ 32.5mm}{88.5mm}{\frtnrm Keywords}
    \XYBC{32.5mm}{78.5mm}{162.5mm}{88.5mm}{\tenrm\tenrm\baselineskip=3.5mm%
	\begin{tabular}{p{127.5mm}}\@英文キーワード\end{tabular}
    \baselineskip = 4.0mm plus1mm minus0.5mm}

    \XYBC{ 2.5mm}{ 88.5mm}{ 32.5mm}{100.5mm}{\svtnrm Summary}
    \XYBP{12.5mm}{100.5mm}{154.5mm}{243.5mm}{\twlrm \the\英文論文要旨}
  \fi
%
  \def\baselinestretch{1.5}\large\normalsize
  \vfil\break
}}
%</shuron>
%    \end{macrocode}
% \end{macro}
%
% \section{ils フォント定義}
%    \begin{macrocode}
%<*fonts>
\DeclareFontFamily{U}{ils}{}
\DeclareFontShape{U}{ils}{m}{n}{
  <-> ils}{}
\DeclareTextSymbol{\k}{U}{65}
\DeclareTextSymbol{\K}{U}{66}
\DeclareTextSymbol{\wda}{U}{67}
\DeclareTextSymbol{\bda}{U}{68}
\DeclareTextSymbol{\wra}{U}{69}
\DeclareTextSymbol{\bra}{U}{70}

%% ここから先はシンボル表示のマクロ
%%%%%%%%%%%%%%%%%%%%%%%%%%%%%%%%%%%%%%%%%%%%%%%%%%%
%% 工大マーク (小)
\def\koudai{%
  \edef\tmp@encoding{\f@encoding}%
  \edef\tmp@family{\f@family}%
  \fontencoding{U}%
  \fontfamily{ils}%
  \selectfont%
  \k%
  \fontencoding{\tmp@encoding}%
  \fontfamily{\tmp@family}%
  \selectfont%
}
%% 工大マーク (大)%
\def\Koudai{%
  \edef\tmp@encoding{\f@encoding}%
  \edef\tmp@family{\f@family}%
  \fontencoding{U}%
  \fontfamily{ils}%
  \selectfont%
  \K%
  \fontencoding{\tmp@encoding}%
  \fontfamily{\tmp@family}%
  \selectfont%
}
%% 白抜き下矢印%
\def\whitedownarrow{%
  \edef\tmp@encoding{\f@encoding}%
  \edef\tmp@family{\f@family}%
  \fontencoding{U}%
  \fontfamily{ils}%
  \selectfont%
  \wda%
  \fontencoding{\tmp@encoding}%
  \fontfamily{\tmp@family}%
  \selectfont%
}
%% 黒塗り上矢印%
\def\blackdownarrow{%
  \edef\tmp@encoding{\f@encoding}%
  \edef\tmp@family{\f@family}%
  \fontencoding{U}%
  \fontfamily{ils}%
  \selectfont%
  \bda%
  \fontencoding{\tmp@encoding}%
  \fontfamily{\tmp@family}%
  \selectfont%
}
%% 白抜き右矢印%
\def\whiterightarrow{%
  \edef\tmp@encoding{\f@encoding}%
  \edef\tmp@family{\f@family}%
  \fontencoding{U}%
  \fontfamily{ils}%
  \selectfont%
  \wra%
  \fontencoding{\tmp@encoding}%
  \fontfamily{\tmp@family}%
  \selectfont%
}
%% 黒塗り右矢印x%
\def\blackrightarrow{%
  \edef\tmp@encoding{\f@encoding}%
  \edef\tmp@family{\f@family}%
  \fontencoding{U}%
  \fontfamily{ils}%
  \selectfont%
  \bra%
  \fontencoding{\tmp@encoding}%
  \fontfamily{\tmp@family}%
  \selectfont%
}
%</fonts>
%    \end{macrocode}
%
% \section{ils フォント定義}
%    \begin{macrocode}
%<*ilsmf>

%  This is METAFONT FILE . Chiba Institute of Technology
%  Computer Science Itakura Lab. 9010116 Akira Yoshizawa.
%  Create:1993-09-19
%
%  And after, modified and added by Mippy.
%
%
%  CIT sigh char No 65
%
mode_setup; font_identifier:="ILS"; font_size = 10pt#;
%
%   wid#:=18mm#;% <- org.
%   hig#:=21mm#;% <- org.
   wid#:=2.3mm#;%           <- 1/7.82608695652*org
   hig#:=2.683333333mm#;%   <- 1/7.82608695652*org
   dip#:=0;
%
%
beginchar(65,wid#,hig#,dip#);
%
   numeric penA,penB,penC,penD;
%
%
   penA = w/9;
   penB = w/6;
   penC = 2w/9;
   penD = 5w/18;

%
z99 = (     0,  w/12);
z1  = (  w/12,41h/42) - z99;
z2  = (11w/12,41h/42) - z99;
z3  = (  w/ 2,39h/42) - z99;
z4  = (  w/ 2,33h/42) - z99;
z5  = (13w/36, 6h/ 7) - z99;
z6  = (23w/36, 6h/ 7) - z99;
z7  = (  w/12,31h/42) - z99;
z8  = (11w/12,31h/42) - z99;
z9  = (  w/ 2, 9h/14) - z99;
z10 = (  w/ 2,23h/42) - z99;
z11 = (  w/12,  h/ 2) - z99;
z12 = (11w/12,  h/ 2) - z99;
z13 = (  w/12, 7h/42) - z99;
z14 = (11w/12, 7h/42) - z99;
z15 = ( 2w/ 9,17h/42) - z99;
z16 = ( 7w/ 9,17h/42) - z99;
z17 = (13w/36, 3h/ 7) - z99;
z18 = (13w/36, 4h/21) - z99;
z19 = (23w/36, 3h/ 7) - z99;
z20 = (23w/36, 4h/21) - z99;
z21 = (  w/ 2,17h/42) - z99;

z22 = ( 5w/18, 6h/ 7) - z99;
z23 = (13w/18, 6h/ 7) - z99;
z24 = (  w/ 2,13h/21) - z99;
z25 = (  w/ 2,  h/ 2) - z99;
z26 = ( 2w/ 9, 8h/21) - z99;
z27 = (  w/ 2, 8h/21) - z99;
z28 = ( 7w/ 9, 8h/21) - z99;

pickup pencircle scaled penD;
draw z3--z4;
pickup pencircle scaled penB;
draw z1--z2;
draw z7--z8;
pickup pencircle scaled penC;
undraw z9--z24;
undraw z9--z24;
draw z10--z25;
draw z10--z25;
draw z10--z25;
draw z10--z25;
pickup pencircle scaled penB;
draw z11--z12;
draw z11--z13;
draw z12--z14;
draw z17--z18;
draw z19--z20;
pickup pencircle scaled penA;
undraw z5--z22;
undraw z6--z23;
undraw z15--z26;
undraw z21--z27;
undraw z21--z27;
undraw z21--z27;
undraw z21--z27;
undraw z21--z27;
undraw z16--z28;
labels(1,2,3,4,5,6,7,8,9,10,11,12,13,14,15,16,17,18,19,20,21);
endchar;
%
%  CIT sigh char No 66   、チ、遉テ、ネツ遉ュ、盪ゥツ逾゛。シ・ッ
%
%mode_setup; font_identifier:="ILS"; font_size = 10pt#;
%
%   wid#:=18mm#;% <- org.
%   hig#:=21mm#;% <- org.
   wid#:=6mm#;% <- 1/3*org
   hig#:=7mm#;% <- 1/3*org
   dip#:=0;
%
%
beginchar(66,wid#,hig#,dip#);
%
   numeric penA,penB,penC,penD;
%
%
   penA = w/9;
   penB = w/6;
   penC = 2w/9;
   penD = 5w/18;
%
z99 = (     0,  w/12);
z1  = (  w/12,41h/42) - z99;
z2  = (11w/12,41h/42) - z99;
z3  = (  w/ 2,39h/42) - z99;
z4  = (  w/ 2,33h/42) - z99;
z5  = (13w/36, 6h/ 7) - z99;
z6  = (23w/36, 6h/ 7) - z99;
z7  = (  w/12,31h/42) - z99;
z8  = (11w/12,31h/42) - z99;
z9  = (  w/ 2, 9h/14) - z99;
z10 = (  w/ 2,23h/42) - z99;
z11 = (  w/12,  h/ 2) - z99;
z12 = (11w/12,  h/ 2) - z99;
z13 = (  w/12, 7h/42) - z99;
z14 = (11w/12, 7h/42) - z99;
z15 = ( 2w/ 9,17h/42) - z99;
z16 = ( 7w/ 9,17h/42) - z99;
z17 = (13w/36, 3h/ 7) - z99;
z18 = (13w/36, 4h/21) - z99;
z19 = (23w/36, 3h/ 7) - z99;
z20 = (23w/36, 4h/21) - z99;
z21 = (  w/ 2,17h/42) - z99;

z22 = ( 5w/18, 6h/ 7) - z99;
z23 = (13w/18, 6h/ 7) - z99;
z24 = (  w/ 2,13h/21) - z99;
z25 = (  w/ 2,  h/ 2) - z99;
z26 = ( 2w/ 9, 8h/21) - z99;
z27 = (  w/ 2, 8h/21) - z99;
z28 = ( 7w/ 9, 8h/21) - z99;

pickup pencircle scaled penD;
draw z3--z4;
pickup pencircle scaled penB;
draw z1--z2;
draw z7--z8;
pickup pencircle scaled penC;
undraw z9--z24;
undraw z9--z24;
draw z10--z25;
draw z10--z25;
draw z10--z25;
draw z10--z25;
pickup pencircle scaled penB;
draw z11--z12;
draw z11--z13;
draw z12--z14;
draw z17--z18;
draw z19--z20;
pickup pencircle scaled penA;
undraw z5--z22;
undraw z6--z23;
undraw z15--z26;
undraw z21--z27;
undraw z21--z27;
undraw z21--z27;
undraw z21--z27;
undraw z21--z27;
undraw z16--z28;
labels(1,2,3,4,5,6,7,8,9,10,11,12,13,14,15,16,17,18,19,20,21);
endchar;
%
%
%
%  ----- downarrow sigh char No 67 -----
%
%mode_setup; font_identifier:="ILS"; font_size = 10pt#;
%
   wid#:=15mm#;
   hig#:=17.5mm#;
   dip#:=0;
%
%
%
beginchar(67,wid#,hig#,dip#);
%
%
   z1 = (w/3,h);
   z2 = (w/6,3w/7);
   z3 = (w/2,0);
   z4 = (5w/6,3w/7);
   z5 = (2w/3,h);
   z6 = (w/3,3w/7);
   z7 = (2w/3,3w/7);
%
   pickup pencircle scaled 0.3pt;
   draw z1--z6--z2--z3--z4--z7--z5--cycle;
%
endchar;
%
%
%  ----- downarrow sigh char No 68 -----
%
beginchar(68,wid#,hig#,dip#);
%
%
   z1 = (w/3,h);
   z2 = (w/6,3w/7);
   z3 = (w/2,0);
   z4 = (5w/6,3w/7);
   z5 = (2w/3,h);
   z6 = (w/3,3w/7);
   z7 = (2w/3,3w/7);
%
   fill z1--z6--z2--z3--z4--z7--z5--cycle;
%
endchar;
%
%
   wid#:=17.5mm#;
   hig#:=15mm#;
   dip#:=0;
%
%
%  ----- w-rightarrow sigh char No 69 -----
%
beginchar(69,wid#,hig#,dip#);
%
%
   z1 = (0,h/3);
   z2 = (4h/7,h/6);
   z3 = (w,h/2);
   z4 = (4h/7,5h/6);
   z5 = (0,2h/3);
   z6 = (4h/7,h/3);
   z7 = (4h/7,2h/3);
%
   pickup pencircle scaled 0.3pt;
   draw z1--z6--z2--z3--z4--z7--z5--cycle;
%
endchar;
%
%
%  ----- b-rightarrow sigh char No 70 -----
%
beginchar(70,wid#,hig#,dip#);
%
%
   z1 = (0,h/3);
   z2 = (4h/7,h/6);
   z3 = (w,h/2);
   z4 = (4h/7,5h/6);
   z5 = (0,2h/3);
   z6 = (4h/7,h/3);
   z7 = (4h/7,2h/3);
%
   pickup pencircle scaled 0.3pt;
   fill z1--z6--z2--z3--z4--z7--z5--cycle;
%
endchar;

%
end

bye.
%</ilsmf>
%    \end{macrocode}
%
% \Finale
%
\endinput
