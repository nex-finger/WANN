\section{ニューラルネットワーク}
\subsection{ニューラル素子}
人間を含む生命の脳を構成する神経細胞はニューロンと呼ばれ,人間の脳には140億個のニューロンがありそれぞれのニューロンは平均約8,000個のシナプスを持つとされている.ニューラルネットワークのノードはニューロンをモデルとし,計算機上でニューロンをシミュレートできるよう設計されている.
\\ \textbf{ここに図} \\
ここでは,素子への $ N $ 個の入力 $ S_1, S_2, ..., S_N $ に対して各々の重み $ w_1, w_2, ..., w_N $ となっている.この素子は入力からバイアス $ b $ を足した値を活性化関数 $ f $ の入力とし,活性化関数の出力をノードの出力 $ z $ とする.

\begin{equation}
    z = f(\sum_{i=1}^N S_i + b)
\end{equation}

主に活性化関数 $ f $ には次のような関数を用いる.

\begin{enumerate}
    \item tanh関数
    \begin{equation}
        f(x) = \frac{e^{x} - e^{-x}}{e^{x} + e^{-x}}
    \end{equation}

    \item ReLU関数
    \begin{equation}
        f(x) = 
        \begin{cases}
        x & (x > 0)\\
        0 & (x \leq 0)
        \end{cases}
    \end{equation}
\end{enumerate}