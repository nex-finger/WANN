\section{遺伝的アルゴリズム}
遺伝的アルゴリズム(GA)とは,適用範囲の非常に広い,生物の進化を模倣した学習的アルゴリズムである\cite{遺伝的アルゴリズム}.すなわち何万年,何億年もかけて生物の特徴が進化してきたような遺伝的法則を光学的な手法にモデル化し,また参考にしてタスクを解くものである.自然界における生物の進化過程においては,ある世代を形成している個体の集合,すなわち個体群の中で環境に適している個体が高い確率で生き残り,生き残った個体は交叉や突然変異によって次の世代の個体群が形成される.

\subsection{本論文における基本的動作}
本論文において,新しく個体を形成する際に交叉は使用しない.以降本論文で用いる遺伝的アルゴリズム変則GAと呼ぶ.図は変則GAの動作を流れを表している.

ここに図

\begin{enumerate}
    \item 初期化
    ランダムな性質を持つ個体を $ N $ 個生成し,初期世代の個体群を設定する.

    \item 評価
    各個体についてタスクを解かせ,その進捗により適応度を測定する.

    \item 終了判定
    終了条件を満たしていれば,その時に得られる最適個体を問題の準最適解として出力する.

    \item 再生
    各個体に対応する適応度により並べ替え,適応度の低い個体は淘汰され,適応度の高い個体は増殖する.

    \item 変異
    設定された突然変異の設定により変異を行い,新しい個体群を生成する.変異を行った後の個体群は次世代の個体群として再度評価される.
\end{enumerate}

終了判定には最良個体の適応度や個体群の平均の適応度を参照する場合が多いが,本論文ではあらかじめ設定した世代数を超えたときにのみ終了判定が真になり,プログラムは終了する.また,変異をする過程でエリート保存選択を実行する.これは非常に優れた個体は変異をする前の状態のほうが変異をした後の状態よりも優れている見込みが大きいことから,むやみに変異をしてエリート個体の遺伝子を破壊しないことにつながる.
